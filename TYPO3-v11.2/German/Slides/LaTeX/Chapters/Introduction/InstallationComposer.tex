% ------------------------------------------------------------------------------
% TYPO3 Version 11.2 - What's New (German Version)
%
% @license	Creative Commons BY-NC-SA 3.0
% @link		https://typo3.org/help/documentation/whats-new/
% @language	German
% ------------------------------------------------------------------------------
% Installation using Composer

\begin{frame}[fragile]
	\frametitle{Installation und Upgrade}
	\framesubtitle{Installation mit \texttt{Composer}}

	% decrease font size for code listing
	\lstset{basicstyle=\fontsize{8}{10}\ttfamily}

	\begin{itemize}
		\item Installation mit \href{https://getcomposer.org}{PHP Composer} unter Linux, macOS, and Windows 10:
\begin{lstlisting}
$ cd /var/www/site/
$ composer create-project typo3/cms-base-distribution:^11 typo3v11
\end{lstlisting}

		\item Alternativ kann man eine benutzerdefinierte \texttt{composer.json} erstellen und ausführen:
\begin{lstlisting}
$ composer install
\end{lstlisting}

		\item Das Online-Tool \href{https://get.typo3.org/misc/composer/helper}{Composer Helper}
			macht die Paketauswahl einfacher.

		\item Für weitere Details besuchen Sie:
			\href{https://docs.typo3.org/m/typo3/guide-installation/master/en-us/}{Installation and Upgrade Guide}.\newline

		\item \textbf{Die Verwendung von Composer ist die empfohlene Installationsmethode}

	\end{itemize}
\end{frame}

% ------------------------------------------------------------------------------
